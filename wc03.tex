\documentclass[a4paper]{exam}

\usepackage{amsmath,amssymb, amsthm}
\usepackage[a4paper]{geometry}
\usepackage{hyperref}
\usepackage{mdframed}
\usepackage{arabtex}
\usepackage{utf8}
\setcode{utf8}


\title{Weekly Challenge 03: Closure of Regular Languages}
\author{CS 212 Nature of Computation\\Habib University}
\date{Fall 2024}

\qformat{{\large\bf \thequestion. \thequestiontitle}\hfill}
\boxedpoints

\newcommand{\m}{\text{\<م>}}

% \printanswers %uncomment this

\begin{document}
\maketitle

\begin{questions}
  
\titledquestion{\<زبانوں کا میلاپ>}
    You and your friend Jerry Smith has finally made an interdimensional portal to the Regular world. In Regular world people only communicate with Regular languages. To make sure you're able to survive the Regular world, you have both prepared a regular language, that you will use to communicate with the inhabitants of this world.

    Even after all your precautions, while crossing the portal something went wrong, both you and your friend Jerry got fused together. Due to this the regular languages you prepared also got scrambled together. The new language that you speak only consists of words common in both your initial languages. You being well aware of the Regular world lore know that such a fusing of two beings is known as ``Milaap'' (\<مِلاپ>). The Milaap operator $\m$ is quite a popular operator in Regular world, and you know that it is defined as follows:

    For languages $L_1 \subseteq \Sigma^*$ and $L_2 \subseteq \Sigma^*$ for some set of alphabets $\Sigma$, their Milaap is defined as:
    $$L_1 \m L_2 = \{w \in \Sigma^* | \; w\in L_1 \text{ and } w\in L_2\}$$

    Now worried that your new language after the Milaap (\<م>) of your both's initial Regular languages is still regular. Try to prove/disprove that the class of Regular languages is closed under Milaap operator (\<م>). Argue briefly on why your solution/proof/construction is correct.

    % Enter your solution below
  \begin{solution}

  \end{solution}
  
\end{questions}
\end{document}

%%% Local Variables:
%%% mode: latex
%%% TeX-master: t
%%% End:
